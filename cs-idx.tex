\documentclass{article}

\begin{document}

\title{CS index}
\author{e.v.eremin}

\maketitle

\begin{abstract}
Computer science topics index.
\end{abstract}

\section{Introduction}

What should be known.

\section{Hardware}

\subsection{Central Processing Unit (CPU)}

\subsection{Instruction Execution}

\subsection{Stack}

\begin{itemize}
\item Stack pointer
\item Frame pointer
\item Subroutine/Subprogram
\item Return Address
\end{itemize}

\subsubsection{Stack pointer}

\subsubsection{Frame pointer}

\section{Multiprocessing}

\subsection{TLS}

\subsection{Subprograms/Routines/Functions}

\begin{itemize}
\item Static
\item Thread-safe
\item Re-entrant
\end{itemize}

\section{Locking}

\begin{itemize}
\item futex
\item mutex
\item spinlock
\item rwlock (Reader/Writer Lock)
\item RCU
\end{itemize}

\subsection{RCU (Read, Copy, Update)}

RCU --- Read, Copy, Update.

Paul McKenney series on LWN.net:

\begin{enumerate}

\item What is RCU, Fundamentally?

	\texttt{https://lwn.net/Articles/262464/}

\item What is RCU? Part 2: Usage

	\texttt{https://lwn.net/Articles/263130/}

\item RCU part 3: the RCU API

	\texttt{https://lwn.net/Articles/264090/}

\item The RCU API, 2019 edition

	\texttt{https://lwn.net/Articles/777036/}

\end{enumerate}

\section{Tracing}

\begin{enumerate}
\item Dynamic Tracing
\item ftrace
\begin{verbatim}
mount -t tracefs tracefs /sys/kernel/tracing
\end{verbatim}
\item KProbes
\item Perf
\end{enumerate}

\section{Virtualization}

\subsection{Namespaces}

\begin{enumerate}

\item Separation Anxiety: A Tutorial for Isolating Your System with Linux Namespaces

	\texttt{https://www.toptal.com/linux/separation-anxiety-isolating-your-system-with-linux-namespaces}

\end{enumerate}

\subsection{lxc}

\subsection{lxd}

\subsection{Docker}

\section{Memory Management}

\subsection{Segments}

\subsection{Paging}

\begin{itemize}
\item PGD (Page Global Directory)
\item PUD (Page Upper Directory)
\item PMD (Page Middle (Mid-level) Directory)
\item PTE (Page Table Entry)
\item Page
\end{itemize}

\begin{itemize}
\item PAE (Physical Address Extension)
\item Page Frame.  Physical memory is organized into \emph{page frames}.  The 
size of a page frame is a power of 2 in bytes and varies among systems.
\item Page.  Logical memory is organized into \emph{pages}. The size of page 
matches a page frame.
\item PFN (Page Frame Number).  A PFN is simply an index within physical memory 
that is counted in page-sized units. PFN for a physical address could be 
trivially defined as (\verb"page_phys_addr >> PAGE_SHIFT").
\end{itemize}

\textbf{References}

\begin{enumerate}
\item Four-level page tables merged [Posted January 5, 2005 by corbet]

	\texttt{https://lwn.net/Articles/117749/}

\item Mel Gorman (mel at skynet dot ie).  Understanding the Linux Virtual Memory Manager

	\texttt{https://www.kernel.org/doc/gorman/}
\end{enumerate}

\subsection{Page-replacement algorithm}

\begin{enumerate}

\item Least Recently Used (LRU)

\end{enumerate}

\subsection{References}

\begin{enumerate}

\item CSE 240A: Graduate Computer Architecture

	\begin{enumerate}

	\item Cache (CSE240A-MBT-L15-Cache.ppt.pdf)

	\texttt{https://cseweb.ucsd.edu/classes/fa10/cse240a/pdf/08/}

	\item Virtual Memory (CSE240A-MBT-L18-VirtualMemory.ppt.pdf)

	\texttt{https://cseweb.ucsd.edu/classes/fa10/cse240a/pdf/08/}

	\end{enumerate}

\end{enumerate}

\section{Command Line Interface (CLI)}

\section{Readline}

\begin{verbatim}
	inputrc
\end{verbatim}

\section{Vim}

\begin{verbatim}
	.vimrc
\end{verbatim}

\section{Bash}

\begin{verbatim}
	bashrc
	bash_profile
	profile
\end{verbatim}

\subsection{bashrc}

History.

\begin{verbatim}
HISTCONTROL=ignoreboth
HISTSIZE=100000
HISTFILESIZE=2000000
\end{verbatim}

Vim as the default editor.

\begin{verbatim}
export EDITOR=vim
\end{verbatim}

\begin{verbatim}
export MANWIDTH=80
\end{verbatim}

Git shortcuts.

\begin{verbatim}
alias gil='git log --stat'
alias gig='git grep'
alias gis='git status'
alias gid='git diff'
alias gib='git branch'
alias gia='git add'
alias gic='git commit -v'
alias gif='git log --graph --format="%C(auto)%h %ad %cd %d %ae %s"'
\end{verbatim}

Make \texttt{rm} (remove), \texttt{mv} (move) and \texttt{cp} (copy) commands less dangerous.

\begin{verbatim}
alias rm='rm -i'
alias mv='mv -i'
alias cp='cp -i'
\end{verbatim}

\section{Git}

\begin{verbatim}
# color.decorate.tag
[color "decorate"]
	tag = magenta
\end{verbatim}

\section{Tmux}

\begin{verbatim}
	tmux.conf
\end{verbatim}

\section{Screen}

\begin{verbatim}
	screenrc
\end{verbatim}



\section{Bash}

Configuration files:

\begin{verbatim}
${HOME}/.bashrc
${HOME}/.bash_profile
${HOME}/.profile
/etc/bashrc
/etc/profile
\end{verbatim}

\subsection{Readline}

\begin{verbatim}
inputrc
\end{verbatim}

\subsection{Vim}

\begin{verbatim}
.vimrc
\end{verbatim}

Fix random syntax highlighting breaks

\begin{verbatim}
:syntax sync fromstart
:syntax sync minlines=20
\end{verbatim}

\subsection{Git}

\begin{verbatim}
# color.decorate.tag
[color "decorate"]
	tag = magenta

[user]
	name = E.V.Eremin
	email = e.vl.eremin@yandex.ru
	username = eremin
\end{verbatim}

\subsubsection{Github (or similar)}

Generating an SSH key

\begin{verbatim}
$ ssh-keygen -t rsa -b 4096 -C 'e.vl.eremin@yandex.ru'
$ xclip -sel clip < ~/.ssh/id_rsa.pub
$ git remote add origin 'git@github.com:eremin-ev/cs-idx.git'
\end{verbatim}

\subsection{Tmux}

\begin{verbatim}
	tmux.conf
\end{verbatim}

\subsection{Screen}

\begin{verbatim}
	screenrc
\end{verbatim}

\begin{equation}
    \label{simple_equation}
    \alpha = \sqrt{\beta}
\end{equation}

\section{Conclusion}

Write your conclusion here.

\end{document}

% vim:spell:
